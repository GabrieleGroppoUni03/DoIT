\title{Specifica Supplementare}
\author{Matteo Cervini \and Andrea Cozzi \and Gabriele Groppo \and Daniele Buser}
\date{Scadenza: 11 Gennaio, 2025}

\maketitle

\subsubsection{Nota Introduttiva}
\begin{itemize}
\item \textbf{Disciplina:} Requisiti
\item \textbf{Due:} 11 Gennaio, 2025 1:00 AM (GMT+1) → 11 Gennaio, 2025 7:00 PM (GMT+1)
\item \textbf{Priority:} Medio
\item \textbf{Projects:} DoIT-2
\end{itemize}

\subsubsection{Glossario Tecnico}

\begin{itemize}

\item \sout{\textbf{Account Google}}

\sout{Sistema di autenticazione richiesto per la registrazione nella piattaforma. Fornisce le informazioni base dell'utente come nome, email e foto profilo attraverso il protocollo OAuth.}

\item \textbf{Autenticazione}

Processo di verifica dell'identità dell'utente effettuato tramite \sout{OAuth Google} \underline{email e password}. È un passaggio obbligatorio durante la registrazione che garantisce la sicurezza e l'unicità dell'account.

\item \textbf{Form \sout{Aggiuntivo} \underline{di Registrazione}}

Modulo di registrazione che raccoglie informazioni specifiche del volontario \sout{non disponibili tramite l'account Google}. Include:
\begin{itemize}
    \item Campi obbligatori da compilare
    \item Validazione in tempo reale
\end{itemize}

\item \sout{\textbf{OAuth}}

\sout{Protocollo di autenticazione utilizzato per l'integrazione con Google, che permette:}
\begin{itemize}
    \item \sout{Accesso sicuro ai dati dell'account Google}
    \item \sout{Importazione automatica delle informazioni base}
    \item \sout{Verifica dell'identità dell'utente}
\end{itemize}

\item \textbf{Profilo}

Insieme delle informazioni che caratterizzano il volontario, composto da:
\begin{itemize}
    \item \sout{Dati importati da Google (nome, email, foto)}
    \item Informazioni aggiuntive fornite durante la registrazione
    \item Stato di completezza e verifica
\end{itemize}

\item \textbf{Volontario}

Utente registrato nella piattaforma che:
\begin{itemize}
    \item \sout{Possiede un account Google verificato}
    \item Ha completato il processo di registrazione
    \item Può accedere alle opportunità di volontariato
    \item Può offrire disponibilità per attività non sovrapposte temporalmente
    \item Deve specificare competenze e disponibilità temporali
\end{itemize}
\end{itemize}