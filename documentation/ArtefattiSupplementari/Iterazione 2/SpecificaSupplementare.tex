\title{Specifica Supplementare}
\author{Matteo Cervini \and Andrea Cozzi \and Gabriele Groppo \and Daniele Buser}
\date{Scadenza: 11 Gennaio, 2025}

\maketitle

\subsubsection{Nota Introduttiva}
\begin{itemize}
\item \textbf{Disciplina:} Requisiti
\item \textbf{Due:} 11 Gennaio, 2025 1:00 AM (GMT+1) → 11 Gennaio, 2025 7:00 PM (GMT+1)
\item \textbf{Priority:} Medio
\item \textbf{Projects:} DoIT-2
\end{itemize}

\subsubsection{Requisiti Funzionali}

\paragraph{Registrazione e Gestione degli Utenti}

\begin{enumerate}
    \item Devono essere richieste informazioni obbligatorie per la registrazione, quali:
    \begin{itemize}
        \item Nome e cognome (o ragione sociale per aziende/enti)
        \item Email \sout{e numero di telefono}
        \item Password sicura con requisiti minimi di complessità
    \end{itemize}
\end{enumerate}

\paragraph{Server e Servizi di Hosting}

Il progetto DoIT utilizzerà una soluzione di hosting cloud per garantire scalabilità e affidabilità:

\begin{itemize}
    \item \textbf{Cloud Provider}:
    \begin{itemize}
        \item \sout{Firebase sarà utilizzato come piattaforma principale.}
        \item \underline{Render è usato per hostare il server}
    \end{itemize}
    \item \textbf{Database}: 
    \begin{itemize}
        \item \underline{Neon Database hosta il database remoto}
        \item PostgreSQL per dati strutturati
        \item \sout{Firestore per dati non strutturati}
    \end{itemize}
\end{itemize}

\paragraph{Esigenze di Storage}

Per la gestione dei dati, il sistema utilizzerà:

\begin{itemize}
    \item PostgreSQL per dati strutturati
    \item \sout{Firebase Firestore} \underline{Neon} per messaggistica
    \item \sout{Firebase Storage} \underline{Neon} per file utente
\end{itemize}