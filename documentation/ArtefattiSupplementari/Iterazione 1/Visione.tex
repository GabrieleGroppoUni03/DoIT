\title{Specifica Supplementare}
\author{Matteo Cervini \and Andrea Cozzi \and Gabriele Groppo \and Daniele Buser}
\date{Scadenza: 27 Dicembre, 2024}

\maketitle

\subsubsection{Nota Introduttiva}
\begin{itemize}
\item \textbf{Disciplina:} Requisiti
\item \textbf{Due:} 26 Dicembre, 2024 1:00 AM (GMT+1) → 27 Dicembre, 2024 7:00 PM (GMT+1)
\item \textbf{Priority:} Medio
\item \textbf{Projects:} DoIT-1
\end{itemize}

\paragraph{Scopo del Documento}
Il presente documento descrive la visione per il progetto \textbf{DoIT}, una piattaforma innovativa dedicata alla gestione delle richieste e delle offerte di volontariato. L'obiettivo principale è facilitare la connessione tra chi offre il proprio tempo e chi necessita di supporto, con un focus iniziale sulla città di Milano e, successivamente, sull'intera Lombardia. Il progetto è ispirato all'app ``Attivati'', già operativa nel Trentino, ma mira a proporre un'evoluzione adattata al contesto lombardo, introducendo nuove funzionalità tecniche avanzate e rispondendo alle specifiche necessità locali.

\paragraph{Descrizione del Problema}
\begin{itemize}
\item \textbf{Contesto attuale}: La gestione del volontariato soffre spesso di una mancata connessione tra chi offre aiuto e chi lo richiede. Le opportunità di volontariato non sempre vengono adeguatamente pubblicizzate o rese accessibili, creando un disallineamento tra domanda e offerta.
\item \textbf{Necessità principali}: Vi è il bisogno di una piattaforma centralizzata e user-friendly che consenta di gestire in modo efficace le richieste e le offerte di volontariato.
\item \textbf{Benefici attesi}: Un aumento della partecipazione al volontariato, una migliore distribuzione delle risorse e una maggiore trasparenza e sicurezza per tutti i partecipanti.
\end{itemize}

\paragraph{Stakeholder}
\begin{itemize}
\item \textbf{Cittadini}: Persone disposte a offrire il proprio tempo per il volontariato.
\item \textbf{Associazioni di volontariato}: Organizzazioni che coordinano e gestiscono attività di volontariato.
\item \textbf{Aziende ed enti}: Realtà interessate a promuovere il volontariato aziendale.
\item \textbf{Beneficiari}: Persone o gruppi che necessitano di supporto.
\item \textbf{Comitato di controllo}: Team responsabile di verificare la veridicità delle informazioni e garantire la sicurezza della piattaforma.
\end{itemize}

\paragraph{Obiettivi del Sistema}
\begin{enumerate}
\item Creare un ambiente digitale sicuro e intuitivo per la gestione del volontariato.
\item Facilitare il matching tra richieste e offerte di volontariato.
\item Garantire la sicurezza e la qualità delle interazioni tra gli utenti.
\item Promuovere il volontariato come pratica sociale attraverso l'uso di tecnologie avanzate.
\end{enumerate}

\paragraph{Funzionalità Principali}
\begin{enumerate}
\item \textbf{Registrazione degli utenti}:
    \begin{itemize}
    \item Creazione di profili per volontari, associazioni, aziende ed enti.
    \item Inserimento di informazioni su disponibilità, competenze e preferenze.
    \end{itemize}
\item \textbf{Gestione delle richieste di volontariato}:
    \begin{itemize}
    \item Inserimento di richieste specifiche (es. supporto per anziani, studenti, ecc.).
    \item Dettaglio di durata, orari e posizione geografica.
    \item Descrizione dettagliata delle necessità.
    \end{itemize}
\item \textbf{Gestione delle offerte di volontariato}:
    \begin{itemize}
    \item Inserimento di disponibilità di tempo e competenze.
    \item Localizzazione della disponibilità e delle aree di intervento.
    \end{itemize}
\item \textbf{Sistema di abbinamento e ricerca}:
    \begin{itemize}
    \item Algoritmi di matching basati su categorie, orari e posizione geografica.
    \item Filtri per ricerca avanzata.
    \end{itemize}
\item \textbf{Comitato di controllo e validazione}:
    \begin{itemize}
    \item Verifica delle richieste e delle offerte.
    \item Segnalazione di contenuti inappropriati.
    \end{itemize}
\item \textbf{Sistema di messaggistica integrato}:
    \begin{itemize}
    \item Chat per comunicazioni dirette tra volontari e richiedenti.
    \item Notifiche in tempo reale.
    \end{itemize}
\item \textbf{Sistema di valutazione e feedback}:
    \begin{itemize}
    \item Recensioni degli utenti per garantire qualità e affidabilità.
    \end{itemize}
\item \textbf{Integrazione con social media}:
    \begin{itemize}
    \item Condivisione delle attività e delle disponibilità sui social network.
    \end{itemize}
\end{enumerate}

\paragraph{Ambito del Sistema}
\begin{itemize}
\item \textbf{Incluso}:
    \begin{itemize}
    \item Funzionalità principali elencate sopra.
    \item Focus iniziale su Milano con espansione successiva alla Lombardia.
    \end{itemize}
\item \textbf{Escluso}:
    \begin{itemize}
    \item Gestione di richieste fuori dal territorio lombardo (nella fase iniziale).
    \item Supporto per lingue diverse dall'italiano.
    \end{itemize}
\end{itemize}

\paragraph{Vincoli}
\begin{itemize}
\item \textbf{Tecnologici}: L'app deve essere compatibile con i principali sistemi operativi (iOS e Android).
\item \textbf{Economici}: Rispetto del budget assegnato per lo sviluppo e la manutenzione ossia 0.
\item \textbf{Temporali}: Lancio della prima versione entro 1 mese dall'inizio dello sviluppo.
\end{itemize}

\paragraph{Rischi e Assunzioni}
\begin{itemize}
\item \textbf{Rischi}:
    \begin{itemize}
    \item Mancata partecipazione degli utenti.
    \item Problemi tecnici durante lo sviluppo.
    \item Difficoltà nel garantire la sicurezza delle interazioni.
    \end{itemize}
\item \textbf{Assunzioni}:
    \begin{itemize}
    \item Gli utenti disporranno di smartphone per accedere all'app.
    \item Esiste una domanda significativa di volontariato nella regione.
    \end{itemize}
\end{itemize}

\paragraph{Conclusione}
Il progetto \textbf{DoIT} si propone come uno strumento essenziale per favorire il volontariato nella regione Lombardia, con particolare attenzione alla città di Milano. Grazie all'integrazione di tecnologie avanzate e a un approccio centrato sugli utenti, l'app rappresenta un passo avanti verso una comunità più solidale e connessa.
