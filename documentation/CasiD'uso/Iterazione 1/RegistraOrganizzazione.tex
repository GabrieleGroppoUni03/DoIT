\title{RegistraOrganizzazione}
\author{Andrea Cozzi}
\date{Scadenza: 27 Dicembre 2024}

\paragraph{Nota Introduttiva}
\begin{itemize}
  \item \textbf{Disciplina:} Requisiti
  \item \textbf{Due:} 26 Dicembre, 2024 1:00 AM (GMT+1) → 27 Dicembre, 2024 7:00 PM (GMT+1)
  \item \textbf{Is blocking:} RicercaAbbinata, SSDRegistraOrganizzazione
  \item \textbf{Parent-task:} Casi d'Uso testuali
  \item \textbf{Priority:} Alta
  \item \textbf{Projects:} DoIT-1
\end{itemize}

Il caso d'uso \textbf{"Registrazione Nuova Organizzazione" (UC-002)} descrive il processo attraverso il quale una nuova organizzazione può registrarsi nella piattaforma. Questo processo è considerato di alta priorità essendo una funzionalità core del sistema, con una frequenza prevista di 20-30 registrazioni mensili.

L'organizzazione inizia il processo dalla schermata iniziale dell'applicazione, dove deve scegliere se accedere come volontario o come organizzazione. Dopo aver selezionato "Organizzazione", il sistema reindirizza all'autenticazione Google, un requisito fondamentale per la registrazione.

Una volta fornite le credenziali Google, il sistema verifica automaticamente se esiste già un account associato nel database. Se l'organizzazione è già registrata, viene effettuato automaticamente il login e mostrata la dashboard organizzazione. Se invece l'organizzazione non è presente nel sistema, viene mostrato un form di registrazione specifico.

Il sistema importa automaticamente le informazioni base come nome, email e foto profilo dall'account Google e le precompila nel form. L'organizzazione deve quindi fornire informazioni aggiuntive come nome organizzazione, indirizzo, partita IVA, numero di telefono, descrizione, tipo di organizzazione e sito web (opzionale). Durante la compilazione, il sistema effettua validazioni in tempo reale. Se ci sono campi obbligatori mancanti o compilati in modo scorretto, il sistema impedisce il completamento della registrazione, mostrando messaggi informativi specifici.

\paragraph{Informazioni Generali}

\textbf{ID Caso d'Uso:} UC-002

\textbf{Nome:} Registrazione Nuova Organizzazione

\textbf{Attore Principale:} Organizzazione

\textbf{Precondizioni:}
\begin{itemize}
    \item Organizzazione non registrata nel sistema
    \item Possesso di un account Google
    \item Dispositivo con connessione internet
\end{itemize}

\textbf{Post-condizioni:}
\begin{itemize}
    \item Nuovo account organizzazione creato
    \item Organizzazione autenticata nel sistema
    \item Profilo completo e verificato
\end{itemize}

\textbf{Priorità:} Alta

\paragraph{Scenario Principale}

\subparagraph{Obiettivo}
Permettere a una nuova organizzazione di registrarsi nella piattaforma utilizzando un account Google e fornendo le informazioni necessarie.

\subparagraph{Flusso Base}
\begin{enumerate}
\item Organizzazione apre l'applicazione
    \begin{itemize}
        \item Sistema: Mostra schermata iniziale
    \end{itemize}
\item Organizzazione seleziona "Organizzazione"
    \begin{itemize}
        \item Sistema: Reindirizza all'autenticazione Google
    \end{itemize}
\item Organizzazione esegue autenticazione con Google
    \begin{itemize}
        \item Sistema: Verifica esistenza account nel database
    \end{itemize}
\item Se l'organizzazione non esiste:
    \begin{itemize}
        \item Sistema: Mostra form registrazione organizzazione con:
            \begin{itemize}
                \item Nome organizzazione
                \item Indirizzo
                \item Tipo di organizzazione
                \item Descrizione
                \item Partita IVA (opzionale)
                \item Sito web (opzionale)
            \end{itemize}
        \item Sistema: Precompila i dati importati da Google (email, foto)
    \end{itemize}
\item Organizzazione compila e invia form
    \begin{itemize}
        \item Sistema: Salva nuovo account organizzazione
        \item Sistema: Effettua login automatico
    \end{itemize}
\item Sistema mostra dashboard organizzazione
    \begin{itemize}
        \item Gestione offerte di volontariato
        \item Riepilogo profilo
    \end{itemize}
\end{enumerate}

\paragraph{Scenari Alternativi}

\subparagraph{Scenario: Organizzazione già registrata}
\begin{enumerate}
\item Dopo l'autenticazione Google, sistema rileva account esistente
\item Sistema effettua login automatico
\item Sistema mostra dashboard organizzazione
\end{enumerate}

\paragraph{Eccezioni e Gestione Errori}

\subparagraph{Errore: Form incompleto}
\begin{itemize}
\item \textbf{Condizione:} Campi obbligatori mancanti
\item \textbf{Azione:} Impossibilità a completare registrazione
\item \textbf{Risultato:} Messaggio informativo su campi mancanti o scorretti
\end{itemize}

\subparagraph{Errore: Autenticazione Google fallita}
\begin{itemize}
\item \textbf{Condizione:} Problema con l'autenticazione Google
\item \textbf{Azione:} Interruzione del processo
\item \textbf{Risultato:} Messaggio di errore e possibilità di riprovare
\end{itemize}

\paragraph{Requisiti Speciali}

\subparagraph{Requisiti Non Funzionali}
\begin{itemize}
\item Performance: Registrazione completabile in max 5 minuti
\item Sicurezza: Crittografia dati sensibili
\item Disponibilità: Sistema accessibile 99.9\% del tempo
\end{itemize}

\subparagraph{Vincoli Tecnici}
\begin{itemize}
\item Integrazione OAuth Google
\item Validazione real-time
\item Storage sicuro dati organizzazione
\item Gestione upload documenti
\end{itemize}

\paragraph{Note Aggiuntive}
\begin{itemize}
\item \textbf{Frequenza:} Media (20-30 registrazioni/mese)
\item \textbf{Criticità:} Alta (funzionalità core del sistema)
\item \textbf{Note implementative:}
    \begin{itemize}
        \item Logging dettagliato errori
        \item Sistema di notifica per amministratori
        \item Backup automatico dati organizzazione
        \item Verifica automatica formato dati inseriti
    \end{itemize}
\end{itemize}