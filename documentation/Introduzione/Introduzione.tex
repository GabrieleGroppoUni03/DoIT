Il progetto DoIT rappresenta una piattaforma innovativa dedicata alla gestione delle richieste e offerte di volontariato. 
L'obiettivo principale è creare un ponte digitale tra volontari e organizzazioni, semplificando il processo di incontro tra 
domanda e offerta di supporto.
Questa iniziativa si pone l'ambizione di migliorare la partecipazione al volontariato e di ottimizzare l'uso delle risorse, 
offrendo un ambiente digitale sicuro, intuitivo e ricco di funzionalità avanzate. Con un focus iniziale sulla città di Milano e
un’espansione futura alla Lombardia, DoIT si ispira a soluzioni già esistenti, ma introduce nuove caratteristiche tecniche e un 
approccio adattato alle esigenze locali.
Il documento esplora i vari aspetti della piattaforma, tra cui \textbf{Requisiti Funzionali} e \textbf{Non Funzionali}, 
\textbf{Casi d'Uso Testuali}, \textbf{Diagrammi} di vari tipo tra i quali: 
    \begin{itemize}
        \item \textbf{Diagrammi dei Casi d'Uso} 
        \item \textbf{Modello delle Classi Concettuali} 
        \item \textbf{Diagrammi di Sequenza di Sistema} 
        \item \textbf{Diagrammi di Interazione, Attivita e di Macchina a Stati} 
        \item \textbf{Diagrammi delle Classi di Progetto}
        \item \textbf{Diagramma dell'Architettura Logica} 
    \end{itemize}
e \textbf{Contratti}. 
Inoltre sono presenti i documenti supplementari della disciplina dei \textbf{Requisiti} che offrono informazioni piu dettagliate, 
sono quindi presenti la \textbf{Specifica Supplementare}, la \textbf{Visione} e il \textbf{Glossario}. 
Questo approccio sistematico garantisce un'\textbf{Analisi} e una \textbf{Progettazione} chiara e orientata all'utente, con particolare 
attenzione a sicurezza, accessibilità e usabilità.
Ogni documento presenta la seguente nota introduttiva:
\begin{itemize}
    \item \textbf{Assigned} Nome Cognome
    \item \textbf{Blocked by} Disciplina, Task oppure insieme di Task
    \item \textbf{Disciplina} Disciplina
    \item \textbf{Due} giorno Mese, anno ora AM/PM (GMT+1) → giorno Mese, anno ora AM/PM (GMT+1)
    \item \textbf{Is blocking} Disciplina, Task oppure insieme di Task
    \item \textbf{Parent-task} Task
    \item \textbf{Priority} Alta/Media oppure Bassa
    \item \textbf{Projects} L'iterazione dell'artefatto, ad esempio DoIT-1 oppure DoIT-2
\end{itemize}
Assigned negli \textbf{Artefatti} è il nome e cognome della persona a cui è assegnato l'\textbf{Artefatto}, al contrario nei 
\textbf{Diagrammi delle Classi Software di Progetto} Assigned è il nome e cognome delle persone che hanno lavorato al codice
di quella parte del progetto. Tutti i \textbf{Diagrammi delle Classi} sono stati realizzati con il sito \textbf{Lucidcharts} e
i diagrammi di: \textbf{Sequenza di Sistema, d'Interazione, d'Attività e di Macchina a Stati} sono stati realizzati con il 
linguaggio di MarkDown \textbf{Mermeid}.
Ogni iterazione è divisa in \textbf{Diagramma di Gantt}, documenti della disciplina dei \textbf{Requisiti}, documenti della
disciplina della \textbf{Progettazione}, i \textbf{Pattern} utilizzati e infine una piccola analisi con \textbf{Understand}.